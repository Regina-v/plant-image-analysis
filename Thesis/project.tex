\documentclass[paper=A4,bibliography=totocnumbered]{scrartcl}
\setkomafont{disposition}{\normalcolor\bfseries} %in KOMA headings are default sans serif, with this patch with serifs see scrguide.pdf, 2010-09-14, page 108
\usepackage[dvipsnames]{xcolor}
\usepackage[latin1]{inputenc} % correct coding for windows e.g. with umlauts
\usepackage[T1]{fontenc}
\usepackage{paralist} % compactenum and compactitem
\usepackage[pdftex]{graphicx} % required to insert figures
\usepackage{listings} % format code
\lstset{escapeinside={<@}{@>}}
\usepackage[margin=10pt,font=small,labelfont=bf,labelsep=endash,singlelinecheck=false,format=plain]{caption} % formatting of captions
\usepackage{booktabs} %for professional tables using \toprule, \midrule, \bottomrule
\usepackage[round]{natbib} % http://merkel.zoneo.net/Latex/natbib.php for author-year citation with round brackets
\usepackage{vmargin} % for margins, especially bottommargin smaller, alternative to DIV=12
\setmarginsrb{35mm}{20mm}{25mm}{15mm}{12pt}{11mm}{0pt}{11mm}

% math packages
\usepackage{amsmath}
\usepackage{amssymb}
\usepackage{centernot}
\usepackage{leftidx}

% set up chapter depths
\newcommand{\subsubsubsection}[1]{\paragraph{#1}\mbox{}\\}
\setcounter{secnumdepth}{4}
\setcounter{tocdepth}{4}
\usepackage{subfig}
\usepackage[hidelinks]{hyperref} % linked pdf, necessarily at the end of preamble!!!

%##########################################################
%###################### Help! #############################
%##########################################################
% 
% Strg+T to comment a block 
% 
% \citet --> Mustermann et al. (2042)
% \citet* --> Mustermann, Musterfrau and Musterkind (2042)
% \citep --> (Mustermann et al. 2042)
% \citep* --> (Mustermann, Musterfrau and Musterkind 2042)
% 
% Insert a picture called xxx.png which is stored in the folder pic as a png
%\begin{figure}
%	\centering
%	\includegraphics[width=13cm]{pic/xxx}
%	\caption[Short Title]{Caption insert here}
%	\label{fig:somelabelname}
%\end{figure}
%
%##########################################################
%########### End Preferences, Begin Document ##############
%##########################################################

%opening
\title{Programming Project 03}
\author{Franc�l Lamprecht, Christine Robinson, Regina Wehler}
\date{Winter Semester 2018/19}

\begin{document}

\maketitle

\tableofcontents
\clearpage
\section{Introduction}
This is a dummy sentence that shows how citations work \citep{Adams.2018}.


\section{Methodology}
The image analysis approach consists of three main steps (figure \ref{fig:pipeline}). First, the input RGB input image is segmented into two class, plant and background by a trainable Weka segmentation classifier. Second, the binary class output is used to identify single leaf objects by watershed segmentation. The resulting binary mask is utilized to analyze properties of the leaf objects in the single red, green and blue channel, respectively.

\begin{figure}
	\centering
	\subfloat[Input]{\includegraphics[width=.3\textwidth]{pic/27_rgb}}
	\qquad
	\subfloat[Classification]{\includegraphics[width=.3\textwidth]{pic/27_class}}
	\qquad
	\subfloat[Watershed]{\includegraphics[width=.3\textwidth]{pic/27_watershed}}
	\qquad
	\subfloat[Particle Analysis]{\includegraphics[width=.3\textwidth]{pic/27_greenPA}}
	\caption[Image analysis pipeline]{The analysis of plant leaves is conducted in three steps. The RGB input image (a) is classified into plant and background by a trainable Weka segmentation classifier (b). Single leaf objects are identified by watershed segmentation (c). The binary mask is used to analyze properties in the split red, green and blue channel (d).}
	\label{fig:pipeline}
\end{figure}

\subsection{Segmentation}
The plant is separated from the background with the help of a trainable Weka segmentation classifier, which is available as a plugin for Fiji. Weka provides a GUI to train machine learning algorithms to produce pixel-based segmentations. The user can add traces to classes and train the classifier with those. Afterwards, traces/regions of interest can be adjusted and the classifier can be re-trained to improve classification. Six representative pictures are selected from the 2017 data set for training (plant029, plant145 and plant159 from A1; plant032, plant034 and plant037 from A2). Pictures are chosen to cover the whole range of plant green shades and the range of background characteristics of the given data sets. The Weka Experimenter was used to assess the performance of different machine learning algorithms. Based on these results (table \ref{tab:weka}), FastRandomForest was used as a classifier. By applying the trained classifier to the RGB input images, for each input image a binary classification image is obtained (figure \ref{fig:pipeline}-b). 

\begin{table}[htbp]
	\centering
	\caption{Comparison of classifier performance by Weka Experimenter. The default algorithm parameters were kept.}
	\begin{tabular}{lllllll}
		\toprule
		Algorithm & Percent correct & Precision & Recall & F score & Matthews  & AUC \\
		 & & & & & correlation & \\
		 \midrule
		FastRandomForest & 99.93 & 1.00 & 1.00 & 1.00 & 1.00 & 1.00 \\
		SMO & 92.52 & 0.89 & 0.93 & 0.91 & 0.85 & 0.93 \\
		$k$-nearest Neighbors & 95.38 & 0.95 & 0.93 & 0.94 & 0.90 & 0.95 \\
		RandomSubSpace & 99.85 & 1.00 & 1.00 & 1.00 & 1.00 & 1.00 \\
		Bagging & 99.77 & 1.00 & 1.00 & 1.00 & 1.00 & 1.00 \\
		DesicisonTable & 99.05 & 0.99 & 0.9 & 0.99 & 0.98 & 1.00 \\
		\bottomrule
	\end{tabular}
	\label{tab:weka}
\end{table}

\subsection{Objects Recognition}
A plant consists of leaves that are attached to each other. To be able to analyze leaves individually, they need to be separated. Here, the watershed algorithm was used to separate touching objects. The algorithm first calculates an Euclidean distance map and determines center points as points which are, from a topological view, the ultimate eroded points, As the algorithm's name indicates, this topological map is "flooded" with water and at each collision of two "watersheds", a line separating two objects is drawn. The output of the algorithm is the binary image with added watershed lines (figure \ref{fig:pipeline}-c). 

\subsection{Object Analysis}
Particle Analyzer and additional stuff until csv export

\subsection{Explorative Data Analysis}
Everything in Python

\section{Results}


\section{Discussion \& Outlook}
- watershed is working best for circular objects, that's why it sometimes fails on leaves with long stems



\renewcommand\bibname{References} % necessarily before \bibliography{ref}, but not working in preamble
\bibliographystyle{apalike} % for references chapter
\bibliography{ref}

\end{document}
